\documentclass{report}
\usepackage[a4paper,margin=1in,footskip=0.25in]{geometry}

\input{latex-template/index.tex}

\usepackage{enumerate}

\begin{document}

\chapter{The Reals}

\begin{definition}{Fields}{field}
  A \emph{field} $\left(\F, +, \cdot\right)$ is a nonempty set $\F$, along with two binary operations,
  addition $+: \F \times \F \rightarrow \F$
  and multiplication $\cdot: \F \times \F \rightarrow \F$, satisfying the following,

  \begin{enumerate}[I.]
    \item \textbf{Associativity:} For all $x, y, z \in \F, (x + y) + z = x + (y + z) \tand (x \cdot y) \cdot z = x \cdot (y \cdot z)$.
    \item \textbf{Commutativity:} For all $x, y \in \F$, $x + y = y + x \tand x \cdot y = y \cdot x$.
    \item \textbf{Identities:} There exists elements $0, 1 \in \F$ such that for all $x \in \F, x + 0 = x \tand x \cdot 1 = x$.
    \item \textbf{Inverses:} For all $x \in \F$, there exists an element $-x \in \F$ such that $x + (-x) = 0$, and if $x \neq 0$, there exists an element $x^{-1}$ such that $x \cdot x^{-1} = 1$.
    \item \textbf{Distributive Property:} For all $x, y, z \in \F$, $x \cdot (y + z) = (x \cdot y) + (x \cdot z)$.
  \end{enumerate}
\end{definition}

\begin{notation}{}{}
  Multiplication will usually be written like $xy$ or $(x)(y)$ instead of $x \cdot y$.
  For example the distributive property would be written as $x(y + z) = xy + xz$.
\end{notation}

\begin{notation}{}{}
  Addition of inverse elements will usually be written like $x - y$ instead of $x + (-y)$.
\end{notation}

\begin{lemma}{unique additive identity}{}
  For any field $\left(\F, +, \cdot\right)$ there exists only one additive identity.
\end{lemma}

\begin{proof}
  Assume $\left(\F, +, \cdot\right)$ forms a field, as well assume that $0$ and $0'$ are additive identities in $\F$.
  Recall that this means $x = x + 0 = x + 0'$ for all $x \in \F$.
  Using these identities as well as commutativity we obtain,
  $$0 = 0 + 0' = 0' + 0 = 0'.$$
  This shows that any two additive identities must be the equal, completing the proof.
\end{proof}

\begin{lemma}{unique multiplicative identity}{}
  For any field $\left(\F, +, \cdot\right)$ there exists only one multiplicative identity.
\end{lemma}

\begin{proof}
  Assume $\left(\F, +, \cdot\right)$ forms a field, as well assume that $1$ and $1'$ are additive identities in $\F$.
  Recall that this means $x = 1x =  1'x$ for all $x \in \F$.
  Using these identities as well as commutativity we obtain,
  $$1 = 1 \cdot 1' = 1' \cdot 1 = 1'.$$
  This shows that any two multiplicative identities must be equal, completing the proof.
\end{proof}

\begin{lemma}{$0x = 0$}{}
  For any field $\left(\F, +, \cdot\right)$, $0x = 0$ for all $x \in \F$.
\end{lemma}

\begin{proof}
  Assume $\left(\F, +, \cdot\right)$ forms a field and $x \in \F$.
  Using the additive identity, distributive property and additive inverses we obtain,
  $$0 = x - x = 1x - x = (1 + 0)x - x = 1x + 0x - x = x - x + 0x = 0x.$$
  This shows that $0 = 0x$ for all $x \in \F$, completing the proof.
\end{proof}

\begin{lemma}{$-x = (-1)x$}{}
  For any field $\left(\F, +, \cdot\right)$, $-x = (-1)x$ for any $x \in \F$.
\end{lemma}

\begin{proof}
  Assume $\left(\F, +, \cdot\right)$ forms a field and $x \in \F$.
  Using properties of a field and the previous lemma we obtain,
  $$-x = -x + 0 = -x + 0x = -x + x(1 - 1) = -x + 1x + (-1)x = -x + x + (-1)x = 0 + (-1)x = (-1)x.$$
  This shows that $-x = (-1)x$ for all $x \in \F$, completing the proof.
\end{proof}

\begin{lemma}{$x = -(-x)$}{}
  For any field $\left(\F, +, \cdot\right)$, $-(-x) = x$ for any $x \in \F$.
\end{lemma}

\begin{proof}
  Assume $\left(\F, +, \cdot\right)$ forms a field and $x \in \F$.
  By the definition of a field there exists an element $-x \in \F$ such that $x - x = 0$.
  Furthermore, there exists an element $-(-x) \in \F$ such that $-(-x) - x = 0$.
  Using the properties of a field we obtain,
  $$x = x + 0 = x - x -(-x) = -(-x).$$
  This shows that $x = -(-x)$ for any $x \in \F$, completing the proof.
\end{proof}

\begin{lemma}{$0 = -0$}{}
  For any field $\left(\F, +, \cdot\right)$, $0 = -0$.
\end{lemma}

\begin{proof}
  Assume $\left(\F, +, \cdot\right)$ forms a field.
  Using properties of field as well as previously proven statements we obtain,
  $$0 = 0 \cdot 0 = 0(1 - 1) = (1)0 + (-1)0 = 0 - 0 = -0.$$
\end{proof}

\begin{definition}{}{}
  An \emph{ordered set} $\left(S, >\right)$ is a set $S$ with a relation $>$ called an \emph{ordering} such that,
  \begin{enumerate}[I.]
    \item For all $x, y \in S$ either $x > y$, $y > x$ or $x = y$.
    \item If $x > y$ and $y > z$ then $x > z$.
  \end{enumerate}
  We also define relations $\geq, <$ and $\leq$,
  \begin{enumerate}[I.]
    \item $x \geq y$ if $x > y$ or $x = y$.
    \item $x < y$ if $y > x$.
    \item $x \leq y$ if $x < y$ or $x = y$.
  \end{enumerate}
\end{definition}

\begin{definition}{Ordered Fields}{ordered field}
  A field $\left(\F, +, \cdot\right)$ is an \emph{ordered field} $\left(\F, +, \cdot, > \right)$ if $\left(\F, >\right)$ forms an ordered set satisfying the following,
  \begin{enumerate}[I.]
    \item For all $x,y,z \in \F$, if $x > y$ then $x + z > y + z$.
    \item For all $x,y \in \F$, if $x > 0$ and $y > 0$ then $xy > 0$.
  \end{enumerate}
\end{definition}

\begin{notation}{}{}
  We say a number is \emph{positive} if $x > 0$ and \emph{negative} if $x < 0$.
\end{notation}

\begin{lemma}{$x > y \implies -y > -x$}{}
  If $\left(\F, +, \cdot, >\right)$ is an ordered field, $x, y \in \F$ and $x > y$ then $-y > -x$.
\end{lemma}

\begin{proof}
  Assume that $\left(\F, +, \cdot, >\right)$ is an ordered field, $x, y \in \F$ and $x > y$.
  By the second property of an ordered field $0 > y - x$ furthermore $-y > -x$.
  This completes the proof.
\end{proof}


\begin{definition}{Absolute Value}{abs val}
  Suppose $\left(\F, P, +, \cdot\right)$ is an ordered field, we define the \emph{absolute value} function $|\cdot|: \F \rightarrow \F$ to be,
  $$|x| = \begin{cases}
      x,  & x \geq 0 \\
      -x, & x < 0    \\
    \end{cases}$$
\end{definition}


\begin{proposition}{$|x| > 0$}{}
  If $\left(\F, +, \cdot, >\right)$ is an ordered field,
  then $|x| > 0$ for all $x \in \F$ where $x \neq 0$.
  If $x = 0$ then $|x| = 0$, which is clear from the definition.
\end{proposition}

\begin{proof}
  Assume $\left(\F, +, \cdot, >\right)$ forms an ordered field, $x \in \F$ and $x \neq 0$.
  By the definition of an ordered field $x > 0$ or $x < 0$.

  First consider the case when $x > 0$.
  Notice, by the definition of an ordered set $x \geq 0$ as well.
  By the definition of the absolute value $|x| = x$.
  Thus, $|x| > 0$.

  Now, consider the case when $x < 0$.
  By the definition of the absolute value $|x| = -x$.
  By lemma 1.0.7 $-x > 0$, as well as $|x| > 0$.

  In either case $|x| > 0$ completing the proof.
\end{proof}

\begin{proposition}{$|x| = |-x|$}{}
  If $\left(\F, +, \cdot, >\right)$ is an ordered field,
  then $|x| = |-x|$ for all $x \in \F$.
\end{proposition}

\begin{proof}
  Assume $\left(\F, +, \cdot, >\right)$ is an ordered field.
  Notice that $0 \geq 0$ and thus $|0| = |-0| = 0$.
  There are two other cases $x > 0$ or $x < 0$.
  The argument is the same for either case, so we will just consider the case when $x > 0$.
  By the definition of the absolute value $|x| = x$.
  Using lemma 1.0.7, we know that $-x < 0$.
  By the definition of the absolute value $|-x| = -(-x) = x$.
  This shows that $|x| = |-x|$ for all $x \in \F$, completing the proof.
\end{proof}

\begin{proposition}{$|x| \geq x \geq -|x|$}{}
  If $\left(\F, +, \cdot, >\right)$ is an ordered field,
  then $|x| \geq x \geq -|x|$ for all $x \in \F$.
\end{proposition}

\begin{proof}
  Assume $\left(\F, +, \cdot, >\right)$ is ordered field.
  Notice that this proposition can be split into two, firstly that $|x| \geq x$ and secondly that $x \geq -|x|$.
  We will first prove that $|x| \geq x$ for all $x$ then we will prove that $x \geq -|x|$ for all $x$.

  Assume that $x \in \F$.
  By the definition of an ordered field $x \geq 0$ or $x < 0$.
  First consider the case when $x \geq 0$.
  By the definition of the absolute value $|x| = x$.
  Thus, $|x| \geq x$.
  Now consider the case when $x < 0$.
  Notice that this is equivalent as saying $0 > x$.
  By the definition of the absolute value $|x| = -x$.
  Furthermore, by proposition 1.0.1 $|x| > 0$.
  Applying the transitive property of an ordered field we get that $|x| \geq x$.
  Together both cases show that $|x| \geq 0$ for all $x \in \F$, completing the first half of the proof.

  Just as before, assume that $x \in \F$.
  By the definition of an ordered field $x \geq 0$ or $x < 0$.
  First consider the case when $x \geq 0$.
  By the definition of the absolute value $|x| = x$.
  Thus, $|x| \geq 0$.
  Applying lemmas 1.0.7 and 1.0.6 we obtain that $0 \geq -|x|$.
  Furthermore, applying the transitive property of an ordered field we get that $x \geq -|x|$.
  Now consider the case when $x < 0$.
  By the definition of the absolute value $|x| = -x$.
  Multiplying both sides by $-1$ and applying lemmas 1.0.4 and 1.0.5 we get that $-|x| = x$.
  This shows that $x \geq -|x|$.
  Together both cases show that $x \geq -|x|$ for all $x \in \F$, completing the second half of the proof.
\end{proof}

\begin{proposition}{$|xy| = |x||y|$}{}
  If $\left(\F, +, \cdot, >\right)$ is an ordered field,
  then $|xy| = |x||y|$ for all $x,y \in \F$.
\end{proposition}

\end{document}